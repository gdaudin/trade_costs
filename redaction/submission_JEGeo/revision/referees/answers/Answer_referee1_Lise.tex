\documentclass[a4paper,12pt]{article}

%\usepackage[pdftex]{graphicx}
\usepackage{amsmath}
%\usepackage[latin1]{inputenc}
%\usepackage{hyperref}
%\usepackage[T1]{fontenc}
%\usepackage[utf8]{inputenc}
%%GD Quand je mets usepackage[utf8]{inputenc} je n'arrive plus à compiler la biblio
%J'ai essayé de débuger : je n'y suis pas arrivé
\usepackage{rotating}
\usepackage{setspace}
\usepackage{lscape}
\usepackage[round]{natbib}
\usepackage{multirow}
\usepackage{rotating}
\usepackage{vmargin}
\usepackage{epstopdf}
\usepackage{hyperref}
\usepackage{float}
\usepackage{caption}
\usepackage[tableposition=top]{caption}
\usepackage{amsfonts}
\usepackage{bbold}
\usepackage{bbm}
\usepackage[flushleft]{threeparttable}
%\usepackage{bbold}
%\usepackage[T1]{fontenc}
% JH : impossible de compiler avec le package bbold, remplacé par amsfonts
% LP : impossible de compiler avec le package bbold, remplacé par bbm
\usepackage{upgreek}
\usepackage{comment}
\includecomment{commentGD}
\usepackage[draft]{todonotes}
%\usepackage[disable]{todonotes}
%%GDPour cacher les notes, \usepackage[disable]{todonotes}

\usepackage{setspace}

%\usepackage[nolists, figuresfirst]{endfloat}
%\usepackage{sidefloat}

\setlength{\abovecaptionskip}{-8pt}

%\usepackage[usenames]{color}
%\definecolor{grey}{rgb}{0.35,0.35,0.35}
%\definecolor{webdarkblue}{rgb}{0,0,0.4}
%\definecolor{orange}{rgb}{0.7,0.2,0.05}


%\usepackage[pdfcreator={PDFLaTeX}, pdfproducer={PDFLaTeX}, pdfstartview=FitH, pdfpagemode=UseOutlines, pagebackref=false, colorlinks={true},
%citecolor={webdarkblue}, linkcolor={webdarkblue},
%urlcolor={webdarkblue}]{hyperref}



%\addtolength{\oddsidemargin}{-0.4in}
%\addtolength{\evensidemargin}{-0.4in}
%\addtolength{\textwidth}{0.8in} \addtolength{\topmargin}{-0.85in}
%\addtolength{\textheight}{1.7in}
\renewcommand{\baselinestretch}{1.2}

\newcommand\cites[1]{\citeauthor{#1}'s\ (\citeyear{#1})}

\newcommand\citeh[1]{\citeauthor{#1}'\ (\citeyear{#1})}

\setmarginsrb{3cm}{2cm}{3cm}{2cm}{0,5cm}{0,5cm}{0,3cm}{0,5cm}



% commands

\newcommand{\bi}{\begin{itemize}}
\newcommand{\ei}{\end{itemize}}
\newcommand{\be}{\begin{enumerate}}
\newcommand{\ee}{\end{enumerate}}
\newcommand{\bd}{\begin{description}}
\newcommand{\ed}{\end{description}}
\newcommand{\beqa}{\begin{eqnarray}}
\newcommand{\eeqa}{\end{eqnarray}}
\newcommand{\beq}{\begin{equation}}
\newcommand{\eeq}{\end{equation}}
\newcommand{\bs}{\bigskip}
\newcommand{\A}{$^a$}
\newcommand{\B}{$^b$}
\newcommand{\C}{$^c$}

\renewcommand{\baselinestretch}{1.2}
%\doublespacing

%\linespread{1.6}

%\newcommand{\note}[1]{\footnote{\begin{doublespace}#1\end{doublespace}}}


\begin{document}


\title{\textsc{International Transport costs:\\New Findings from modeling additive costs} \\
Answer to Referee 1}

\author{Guillaume \textsc{Daudin}\thanks{%
Université Paris-Dauphine, PSL University, CNRS, 8007, IRD, 260, LEDa, DIAL, 75016, Paris, France ; email: \url{guillaume.daudin@dauphine.psl.eu}}  \qquad J\'{e}r\^{o}me \textsc{H\'{e}ricourt} \thanks{Universit\'{e} de Lille - LEM-CNRS UMR 9221, France \& CEPII, France; email: \url{jerome.hericourt@univ-lille.fr}}\qquad Lise \textsc{Patureau}\thanks{Corresponding author.
University Paris-Dauphine \& PSL University, LEDa, 75016 Paris, France;  email: \url{lise.patureau@dauphine.psl.eu} } }


\date{September 2020}
 \maketitle
\bigskip

We would like to thank you for your insightful comments. They led us to introduce some
significant changes to the paper that we hope address your concerns. We first give you an overview
of the revision (Section ) before answering in detail each of your comments (Section ). Number of sections, pages and
equations refer to the revised version.

\section{Main changes \label{sec:main_changes}}


\begin{itemize}
\item Based on your comment 1.2 (\textit{The aggregation problem}), the benchmark set-up now retains $s=3$ and $k=10$ for the industry/product aggregation levels respectively. It is true that the original annual Census data reports trade at the origin country-HS 10 product-district level of aggregation. Our initial choice of retaining $s=3$ at the sectoral level, $k=5$ at the product level was driven by the use of Hummel's dataset available over 1974-2004 with $k=5$ the finest degree of aggregation. We yet agree with the referee that jumping from $k=10$ to $k=5$ digits at the product level is detrimental to our ability to exploit useful variation in the data. In line with the referee's comment, we now run the estimation by considering $k$ at the HS10 classification level, not 5. \textbf{We come back to this point in Section XXX of the letter.}

    This has important consequences in terms of database. In the submitted version, we used the original Hummel's database (available on his webpage from 1974 to 2003, which we completed by the US Bureau of census import database over the period 2005-2013. In Hummels' (2007) database, which we use until 2004, the finest degree of classification is at the 5-degree level until 2004. For comparability reasons over time, we have adopted the same degree of classification $k=5$ for the following years (based on the original Census dataset). As a result, in the submitted version, transport costs are estimated at the 3-digit sector level as benchmark estimate, exploiting product heterogeneity at the 5-digit level (for a given origin country, and conditional on year and transport mode).

    Switching to $k=10$ as benchmark disaggregation level for the product dimension requires to resort on US Census dataset only. As first step, we have re-run the estimation over the years for which we already had the data (2005-2013). In parallel, we have bought older years from the Census bureau, as well as the more recent years 2014-2019. However, the Census bureau can provide data disaggregated at the $k=10$ product level only starting in 1989. This deprives us from the very first years 1974-1988. For this period, we can only resort on Hummel's dataset, with $k=5$ the finest product level category. This notably induces us to maintain $s=3$ as the benchmark disaggregation level for the sectoral dimension for the whole period. This allows us to preserve (as far as possible) the largest time coverage - even though switching from $s=3, k=5$ (1974-1988) to $s=3, k=10$) (1989-2019) undoubtedly constitutes a significant break. \textbf{What do we do with that?}



\item Based on your comment 1.1 (\textit{The aggregation problem}), the benchmark regression (non-linear least squares) is now completed by a two-stage procedure, where the fas price is instrumented by tax duties and the lag price \textbf{(See Section XXX).}

\item Based on your main comment 1, the paper has been rewritten so as to be more in line with Hummel's (2004) methodology. Specifically, we now start from the share of additive costs $\beta$, from which we can deduce the additive / ad-valorem components ($t,\tau$) rather than the other way round (as in the submitted version). We thank the referee for pointing this out, as this contributes to a more straightforward and intuitive understanding of our estimation method.

\item What about the big picture?
\end{itemize}

We now answer in detail each of your comments given in italics.

\section{Detailed answers \label{sec:detailed_answers}}
\subsection{Critique 1: Empirical Strategy}

\textit{Using the notation of the authors, they are intreated
in identifying the share of the specific cost in the total transport cost. Namely,}
\begin{eqnarray*}
&& \frac{\frac{t_{is(k)}}{\tilde{p}_{ik}}}{\tau_{is(k)}-1 + \frac{t_{is(k)}}{\tilde{p}_{is(k)}}} \\
\text{or,} &&\frac{t_{is(k)}}{(\tau_{is(k)}-1)\tilde{p}_{is(k)} + t_{is(k)}}
\end{eqnarray*}

\textit{The way they approach the problem is that they assume that (a) $\tau_{ik} = \tau_i\tau_{k}$,
(b) $t_{ik} = t_i +t_k$, and (c) $t_k$ and $\tau_k$ are uniform across products within industry
s. After imposing these assumptions, they estimate the following specification}

\begin{equation}
\ln\left(\frac{p_{ik}}{\widetilde{p}_{ik}}-1 \right)= \ln \left(\tau_{i} \times \tau_{s(k)} -1+\frac{t_{i} + t_{s(k)}}{\widetilde{p}_{ik}} \right) + \epsilon_{ik} \label{eq:equation0}
\end{equation}

\textit{in which $\tau_i$, $\tau_{s(k)}$, $t_i$, and $t_{s(k)}$ are identified as fixed effects coefficients.
In my opinion this choice of strategy is quite sub-optimal, as (i) it relies on
the strong assumptions highlighted above, (ii) it is computationally expensive
as noted by the authors on multiple occasions, and (iii) it is subject to an
endogeneity problem, which the authors disregard with one sentence, but which
is rather detrimental in my opinion.}

The three points raised by the referee indeed deserve careful consideration.

\paragraph{(i) On the assumptions underlining our empirical approach}
Our main empirical equation and its underlying assumptions regarding the separability of transport costs between their country- and product-level components draw on the one proposed by Irarrazabal, Moxnes and Opromolla in the Review of Economics and Statistics (2015) to estimate the share of additive costs in a firm-level context. It relies on a simple theoretical framework with minimal assumptions, and is compatible with most approaches within the so-called category of ``New Trade Theories''.

Further, we already provide a robustness check for this separative assumption, in Section XXX of the paper (\textbf{Table 2, submitted version}). We check the robustness of our results to the separability assumption by re-running the estimation without the separability assumption. This comes at the cost of a substantial increase in the number of fixed effects to include in the regression. Based on the numbers of countries and sectors reported in \textbf{Table 1 of the submitted version} (188 countries, 230 3-digit sectors), it would mean including 86,480 fixed effects ($188\times230\times2$ for the two additive and multiplicative costs) rather than ``only'' 836 ($(188+230)\times 2$). This is not tractable computationally. For this reason, we have decided to run the robustness check on a reduced sample. If it is smaller in terms of observations (2,381 for Air, 2,798 for Vessel on average over the period 1974-2013, vs around 29,000 on the complete sample, see Table 1 ), it yet remains quite large in terms of trade coverage, as the selected countries and sectors (at the 3-digit level) constitute 80\% of the total value of flows (on a yearly basis). As we conclude at the end of Section XXX , whatever the transport mode and for both types of transport costs, the trend patterns of international transport costs are very similar whether estimated under the separability assumption or not. Given that the restricted sample covers 80\% of the total trade flows of the US economy, we hope to convince you that this result can be extended to the whole dataset.


\paragraph{(ii) On the use of non-linear least squares (NLS)}
As highlighted by the referee, our estimated equation imposes to use non-linear estimation methods, such as Non-Linear Least Squares. However, even with another formulation, such as the one suggested by the referee in Equation (\ref{eq:estimation_ref1}), we would still be constrained to resort to non-linear estimators. This is due to the necessity of imposing an \textit{ex-ante} restriction on parameters, i.e. $\tau \geq 1$ and $t \geq 0$, or $0 \leq  \beta \leq 1$. When run relaxing these restrictions, standard linear, least squares estimates often deliver negative trade costs, which is meaningless. In this respect, implementing the referee's method (see below) does not suppress the requirement of resorting on non-linear estimates (and the computational burden it induces, in particular in terms of time). Imposing this parameters constraint was not made clear enough in the initial version, and we did our best to make this very important justification clearer in the revised version \textbf{[see page XX : XX insert quote XX].} \smallskip


Of course, this does not prevent us for taking full account of your suggestion to adopt an alternative estimation method (summarized in Equation (\ref{eq:estimation_ref1}). It turns out that this method faces some important limitations, that justify maintaining our initial estimation strategy as the baseline in the building block of the paper. \textbf{The referee's method is presented in the online appendix, and we refer to it in Footnote XXX of the paper}. We develop the analysis we held in \textbf{Section XXX of the letter}.


\paragraph{(iii) Endogeneity problem}

Indeed, this is a very important point. The referee states that, based on theoretical insights by Melitz (Econometrica, 2003) or Baldwin and Harrigan (AER, 2011), $1/\tilde{p}_{ik}$ is correlated with residuals $\epsilon_{ik}$. In other words, more productive firms and/or firms selling high-quality products will charge higher prices, all other things equal – in our case, for a given country-product pair. We obviously do not question this conceptual issue. However, it is worth noting that a good deal of the bias (actually, the part relating to the quality effect) is going to appear identically in the CIF ($p$) and the FAS ($\tilde{p}$) prices. Consequently, since our dependent variable is based on a ratio between the former and the latter, the (reverse causality) bias cancels out. That said, remains the possibility that bigger firms may impact transport costs, due to their ability of bargaining discounts for larger shipped volumes. Though, we do not know why that should affect differently additive and multiplicative trade costs. Following the referee's advice, we decided therefore to provide some IV-estimates to assess the size of the potential bias. \textbf{This is made in Section XXX of the paper.}

\paragraph{Strategy of answer} To clarify the discussion, we decided to answer concerns (i) and (iii) each in turn. As regards to point (i), we run the referee's proposed estimation method without relying on instrumental variables; this way, the results are comparable with the benchmark results, the only change being the estimation strategy (Section \ref{subsec:functional_form} of the letter). As noted above, this drove us to maintain our estimation method as benchmark. As regards to point (iii), we instrument fas prices at the $k=$ 10-digit level, still based on our original estimation strategy (but switching to a finest product category disaggregation level $k=10$). This allows a direct comparison of the results obtained in the non-linear least squares regression with those obtained with IV (with $s=3$ digit at the sector level and $k=10$ at the product level in both cases). This is developed in \textbf{Section ??? of the letter}.



\subsubsection{Exploring Referee's alternative functional form \label{subsec:functional_form}}


\textit{A more natural approach is what the authors, at some point, refer to as the
Hummel's Methodology. That is, one can alternatively estimate the share of the
additive component as:}

$$\frac{t_{ik}}{ (\tau_{ik}-1)\tilde{p}_{ik} + t_{ik}} = \beta_{ik}$$

\textit{where $\beta_{ik}$ is the elasticity of transport costs w.r.t. unit price} in absolute value, as shown with more details in Appendix \ref{app:interpret_beta} through Equation (\ref{eq:Hummels}). \textit{Given the authors' objective and the data they are using, $\beta$ can be separately estimated
for each industry-country pair using the following regression:}

\begin{equation}
\ln f_{ikd} = \beta_{is(k)}\ln \tilde{p}_{ikd} + \text{Controls}_{ikd} +\epsilon_{ikd} \label{eq:estimation_ref1}
\end{equation}

\textit{where $d$ denotes the US district of entry and $k$ denotes an HS10 product} ($f_{ikd}$ being the transport costs). \textit{The
identification of $\beta_{ik}$, in this case, would rely on the across HS10 product and
district-of-entry variation in $f_{ik}$ and $p_{ik}$. Estimating the above equation would
obviously require that the authors do not aggregate up the raw Census data
across all districts and all 10-digit products pertaining to the same 5-digit category.}


The estimation strategy suggested by the referee starts from Equation (\ref{eq:Hummels}) linking transport costs and unit prices as assumed in Hummels (2007). As shown in Subsection (\ref{ssec:interpret_beta}), the elasticity of transport costs to unit prices $\beta_{ik}$ (in absolute value) also corresponds to the share of additive costs in total transport costs. The share of additive costs can hence be uncovered by regressing transport costs on unit (ie, fas) prices. Specifically, the referee suggests to run the estimation based on Equation (\ref{eq:estimation_ref1}).

Precisely, by year and transport mode, this implies running the estimation for each country of origin $i$ and each sector $s(k)$ = 3 digit sector, exploiting the variability between sub-sectors at the 10-digit level ($k$) and between ports of entry in the US ($d$). We have implemented the referee's method over the years for which we already have the data, i.e. the years 2005-2013 (already used in the submitted version).

Notice that we are aware of the referee's suggestion to run his/her estimation strategy considering sectoral aggregation at the 5-digit level (with $k=10$). We yet keep considering sectors at the 3-digit aggregation level, in view of being able to compare the pros and cons of the referee's method versus ours, including those years for which the finest disaggregation level is $k=5$ (i.e., the original Hummel's dataset \textbf{over 1974-1988, as the US Bureau of census import database starts in 1989.})

Despite its interest, implementing the referee's method uncovered some drawbacks, that in our view, outweighs those of our method (in particular associated with \textit{Assumptions (a) $\tau_{ik} = \tau_i\tau_{k}$, and (b) $t_{ik} = t_i+ t_{k}$.}) We develop our results and conclusion by answering each argument made by the referee at a time.

%
%\paragraph{What we do}
%
%Our estimated equation is written as Equation (\ref{eq:equation0}) with $s$= 3 (or 4) digit and $k$=5 digit (based on Hummel's original dataset from 1974-2004). We thus exploit the variability of transport costs between countries and between 3-digit sectors (conditional on a year and a transport mode, air or vessel). To sum up our methodology:
%\begin{itemize}
%\item Estimating Equation \ref{eq:equation0} provides us with estimates of $\hat{t}_i$, $\hat{t}_{s(k)}$, $\hat{\tau}_i$, $\hat{\tau}_{s(k)}$.
%\item We rebuild
%\begin{eqnarray*}
%\hat{t}_{is(k)} &= &\hat{t}_{i}+\hat{t}_{s(k)} \\
%\hat{\tau}_{is(k)} &= & \hat{\tau}_{i}\times\hat{\tau}_{s(k)}
%\end{eqnarray*}
%\item We deduce the weighted average value of each component $\bar{t}$, $\bar{\tau}$ by year and transport mode, the weighting scheme being the relative value of the flows for each $i,s$ flow, as well as the median, the maximum and the minimum values.
%\item With this in hand, we can (among other things) obtain the estimated share of additive costs in total transport costs for each $i,k$ flow:
%
%
%$$\hat{\beta}_{ik} = \frac{\frac{\hat{t}_{is(k)}}{\tilde{p}_{ik}}}{\tau_{is(k)}-1 + \frac{t_{is(k)}}{\tilde{p}_{is(k)}}} $$
%\end{itemize}
%
%\paragraph{What the referee suggests to do}
%
%The estimation strategy suggested by the referee starts from Equation (\ref{eq:Hummels}) linking transport costs and unit prices as assumed in Hummels (2007). As shown in Subsection (\ref{ssec:interpret_beta}), the elasticity of transport costs to unit prices $\beta_{ik}$ (in absolute value) also corresponds to the share of additive costs in total transport costs. The share of additive costs can hence be uncovered by regressing transport costs on unit (ie, fas) prices. Specifically, the referee suggests to run the estimation based on Equation (\ref{eq:estimation_ref1}).
%
%Precisely, by year and transport mode, this implies running the estimation for each country of origin $i$ and each sector $s(k)$ = 3 digit sector, exploiting the variability between sub-sectors at the 10 digit level ($k$) and between ports of entry in the US ($d$).\footnote{The referee suggests to run his/her proposed estimation strategy considering sectoral aggregation at the 5-digit level (with $k=10$). We followed his suggestion but considering the 3-digit level. We made this choice in view of being able to compare the pros and cons of the referee's method versus ours. In Hummels' (2007) database, which we use until 2004, the finest degree of classification is at the 5-degree level until 2004. For comparability reasons over time, we adopt the same degree of classification $k=5$ for the following years (based on the original Census dataset). As a result, transport costs are estimated at the 3-digit level as benchmark estimate, exploiting heterogeneity at the 5-digit level (for a given origin country, and conditional on year and transport mode). This has driven us to retain $s=3$ when implementing the referee's method, in view of preserving comparability between both strategies.}


\paragraph{a) The advantages of the referee's method are not as important as suggested}


\begin{itemize}
\item \textit{The first advantage of this so-called Hummel's Approach is that the above
regression can be estimated separately for various country-industry pairs, without
imposing Assumptions (a) $\tau_{ik} = \tau_i\tau_{k}$, and (b) $t_{ik} = t_i+ t_{k}$.} Our answer is twofold. 

First, the separability assumption does not seem to be a strong assumption. This is based on the conclusion drawn from the robustness check on a reduced sample (in the submitted version, considering $s=3, k=5$, and in the revised version, considering $s=3, k=10$). If it is smaller in terms of observations,\footnote{In the restricted sample, we have around 25 sectors from 14 countries for Air transport, vs 230 sectors from 190 countries in the large sample; and around 50 sectors from 20 countries for Vessel, vs 600 sectors from 190 countries in the complete sample). Given that the reduced sample makes for 80\% of the total value of trade flows, this suggests that the vast majority of US imports comes from a selected range of countries, in a selected range of sectors.} it yet remains quite large in terms of trade coverage, as the selected countries and sectors (at the 3-digit level) constitute 80\% of the total value of flows (on a yearly basis). As we conclude at the end of Section XXX, whatever the transport mode and for both types of transport costs, the trend patterns of international transport costs are very similar whether estimated under the separability assumption or not. Given that the restricted sample covers 80\% of the total trade flows of the US economy, we hope to convince you that this result can be extended to the whole dataset.


Second, the referee's method imposes a drastic selection of countries / sectors to be included in the estimation. Specifically, since the estimation method is run at the sector-origin country level (on top of being year-mode specific), the number of flows within each country-sector pair should be large enough to more than cover the number of $k$- product and $d$- entry port fixed effects. This substantially reduces the number of point estimates, as well as the value of trade flows covered relative to the  benchmark method, as summarized in Table . 



\item \textit{The second advantage is that there is a handful of [] instruments (e.g., HS-10 product-specific tariff rates or lagged prices), which the authors can use to overcome the endogeneity problem.}  We agree with this point, but we can also handle the instrumentation of fas prices at the HS-10 level with our method - which we do in the revised version. We thank the referee for pointing this out.


\item \textit{The third advantage is that, by adopting this approach, the comparison between the paper [...] and those in Hummels (2007) would become more transparent.} Again, we deeply thank the referee for  \textbf{changer la maniere dont on presente en mettant les beta en avant ?}
\end{itemize}



\paragraph{b) The costs are high}
\begin{itemize}
\item[Concern 1] The suggested estimation strategy implies having much less data to exploit, for two reasons.
\begin{enumerate}
\item Information about the port of entry are only available since 1989. Implementing the referee's method would hence necessarily reduce the time coverage of our analysis by 15 years (1974-1988). In our view, the historical coverage is interesting per se, as it provides useful insights about how transport costs have evolved over time. Eliminating this dimension of the paper would be detrimental to its value-added.
\item As underlined before, the method is run country by country, and 3-digit sector by 3-digit sector, exploiting the variability within each country-3d sector across 10-d sub-sectors and ports of entry. Yet, it appears that for many couples (country, 3-digit sector), there is too few variability across sub-sectors (even at the HS-6 digit classification) or ports en entry given the number of fixed effects included in the regression, such that estimation can not be run. This can be seen comparing the number of observations by year/ transport mode with our method / with the referee's method reported in Table XXX \textbf{TO DO}. Put it differently, this methodology discards countries which export a limited range of goods to the US and/or which arrive in the USA through the same ports of entry. It excludes XXX \% of trade flows and XXX \% of trade in value. In this respect, the induced selection bias reduces the general scope of the transport costs estimates.


    As a direct consequence of this, for things to be comparable across methods, we re-run our estimation strategy on the same sample of observations as the one used in the referee's method.

    \end{enumerate}
\item[Concern 2] The suggested estimation method implicitly assumes that transport costs of say, tissues, have nothing in common whether those goods comes from France or from Bangladesh; or that transport costs that apply to imported goods from France have no common component across sectors; in both cases, one might view this as a disputable assumption. By contrast, we suppose that transport costs are sector-country specific. \textbf{FAIRE REFERENCE A DES PAPIERS EN TRADE QUI ONT DES TRADE COSTS Pays-spécifique?}
\item[Concern 3] The suggested method features less accuracy in the estimation of the $\beta$. If we take the value of the $\beta$ by itself, there is no criteria to discriminate between the value estimated with our method and the one obtained with the suggested method (when, of course, run on the same sample). Things are more clear-cut in terms of accuracy of the estimation. Specifically, our method yields a more precise estimation of the $\beta$ than the referee's method. To develop on this, for each year and transport mode:
    \begin{itemize}
    \item With the referee's method, we estimate one value for the share of additive component $beta$ at the $i,s$ level denoted $\hat{\beta}^{ref}_{is(k)}$ associated with a standard deviation $SD_{is}$. From this, we can approximate the 5- 95\% threshold values through:
        $$\hat{\beta}_{is}^{min,ref} = \hat{\beta}^{ref}_{is(k)} - 1.96 SD_{is},\quad \hat{\beta}_{is}^{max,ref} = \hat{\beta}^{ref}_{is(k)} + 1.96 SD_{is}$$

    \item  With our method, one estimation allows to calculate each transport cost component $\hat{t}_{is}$, $\hat{\tau}_{is}$ and an associated $\hat{\beta}_{is}$. We generate a distribution of $\beta$ through bootstrap method (10,000 random draws), from which we can compute the mean, the median and the 5-95 threshold values for each couple $i,s$.

    \item We can then evaluate the accuracy of each estimation method by comparing the size of the confidence intervals of the $\beta$. This is reported in the Figures XXX (for Air) and XXX (for Vessel). Our method undoubtedly yields more accurate estimations of the share of additive components, whatever the transport mode considered. For sake of brevity, we report the mean distribution averaged over the period, a similar conclusion applies on a yearly basis. \textbf{Averaged over the two transport modes? For only one?} \textbf{GD20200708 Je ne comprends pas bien ce que tu veux dire. Il me semble que les graphiques sont annuels et ne sont pas des moyennes:ce sont des distributions}


\begin{figure}[htbp]
\caption{Accuracy of the estimation of the $\beta$: Comparison}
\label{fig:accuracy_beta}
\begin{center}
\includegraphics[height=4in]{accuracy_beta.pdf}
\end{center}
\end{figure}

Figure \ref{fig:accuracy_beta} reports the log of the interval confidence of the $\beta$ obtained with our methodology (with bootstrap), relative to the one obtained with the referee's method. \textbf{average over the years? the transport modes? unclear}. In most cases, the log of the ratio is negative, implying a lower interval confidence of the $\beta$ estimate with our methodology.

\end{itemize}
\end{itemize}

All these elements drive us to maintain our original estimation in the revised version. \textbf{What do we do with the referee's estimation method? online appendix, Appendix?}
\textbf{GD20200708 Appendix à mon avis. Mais on peut garder la méthode de mesure de précision des betas dans la baseline dans le corps du texte et le remercier abondamment de nous avoir donné l’occasion de discuter ces questions ? ok, appendix du papier plutot}




\subsubsection{Potential endogeneity problem}

\textit{\textbf{The endogeneity problem}: quoting Footnote 14 of the paper, the authors
are estimating ti and ts(k) as coefficients on the industry or country
dummies times $1/\widetilde{p}_{ik}$. [...] Based on the productivity-sorting model in Melitz (2003) or the quality-sorting
model in Baldwin and Harrigan (2010),  $1/\widetilde{p}_{ik}$ is either positively
or negatively correlated with $\epsilon_{ik}$. So, the NNLS estimates are biased; and
the bias has nothing to with the casual versus accounting interpretation of
the estimates. Accordingly, the one-line justification the authors provide
to not address the endogeneity problem is far from convincing.}


We followed the referee's advice by providing some IV-estimates of the fas price to provide a clean assessment of the size of the potential bias in our original methodology. The IV-strategy is implemented in two steps.

\begin{itemize}
\item First-stage equation: We regress the fas price $\widetilde{p}_{ik}$ with $k$ = 5-digit product classification and $i$ the origin country, on tax duty and lagged fas price according to Equation (XXX). As made clear in the appendix of the revised version (\textbf{Section XXX - take elements of topo-first-stage there}), we run the regression in log level on a yearly basis and by transport mode, consistently with our second-stage empirical strategy.

\item The second-stage: Explain the difficulty since fas prices are on both sides of the equation; non-trivial way to do it. We have decided to not instrument the fas price that intervenes on the LHS. Develop.




\end{itemize}

The results are ... DEVELOP.

\subsubsection{The aggregation problem}

\textit{\textbf{The aggregation problem}: The original annual Census data reports
trade at the origin country-HS10 product-district level of aggregation,
whereas the authors are aggregating up the data even further to the origin
country-HS5 industry-year level. Such an aggregation comes with strong implicit assumptions and sacrifices a lot of useful variation in the data.
The authors are motivating the aggregation by stating that the problem
would become computationally expensive without it. But this reasoning
brings us back to my original point that the authors can use the Hummel's
Methodology to circumvent the computational burden.}



It is true that the original annual Census data reports trade at the origin country-HS 10 product-district level of aggregation. Our initial choice of retaining $s=3$ at the sectoral level, $k=5$ at the product level was driven by the use of Hummel's dataset available over 1974-2004 with $k=5$ the finest degree of aggregation. We yet agree with the referee that jumping from $k=10$ to $k=5$ digits at the product level is detrimental to our ability to exploit useful variation in the data. In line with the referee's comment, we now run the estimation by considering $k$ at the HS10 classification level, not 5. After checking with the US Bureau of Census, data is only available since 1989, now available to 2019. We have hence asked for all available data at the time of revision (1989-2009), and we now present the results considering $s=3$ and $k=10$. This however deprives us from being able to exploit the period 1974-1988 at the aggregation level. \textbf{Que faire pour la periode 1974-1989}. Furthermore, the computational burden mentioned by the referee is not attributable to the product classification level $k=5$ or 10) but rather to the degree of sectoral classification ($s=3$ or 4) as it conditions the number of fixed effects. This explains why we consider the $s=4$ digit- sectoral classification level only for some years. We thank the referee for pointing this ambiguity in our paper, which drove us to rewrite the associated paragraph in the revised version of the paper \textbf{see XXXX}



\subsubsection{To sum up: What remains to be done}

\begin{enumerate}
\item\label{bp:3} Re-run our benchmark method with $s=3$ and $k=10$ over available years and compare the results ($t,\tau,\beta$ with our benchmark ones ($s=3,k=5$). Hopefully does not change much so that we can keep up the period 1974-1988? Consider this as our new benchmark. Answer to the critique about the aggregation problem.
\item\label{bp:1} After running the referee's 1 method with $s=3$, $k=$10, show tables or figures comparing 1) the number of observations / countries / sectors 3D by year / transport mode when we do our empirical strategy vs the referee's strategy; 2) comparison of the $\beta$ between the two methods; 3) comparison of the precision of the $\beta$ estimation. Answer to the critique about the estimation strategy.
\item\label{bp:2} IV-First-stage equation on fas price at the $s=3$, $k=10$-digit level, for all years; then 2d-stage estimate of $t,\tau, \beta$ with our methodology; compare with our results obtained in the submitted version. Answer to the critique about the endogeneity bias.
\end{enumerate}

\subsection{Critique 2: Calculation of Unit Prices}

\textit{My second critique concerns the way the authors are calculating the unit prices.
The Census data reports the quantity of goods per observation. So, the authors
can calculate the unit price as Value/Quantity, which is consistent with how
price is modeled in standard trade models. Instead, the authors calculate unit
price as Value/Weight. This used to be a common exercise in the past where
many data-sets did not report Quantity. But, given their data, there is no
justification for the authors to calculate the prices this way.}

The referee is right in the sense that Census data do report the quantity of goods. However, this information is not mode-specific, which is incompatible with our empirical strategy of estimation transport costs conditional on the transport mode, similarly as in Hummels (2004). This explains why we cannot calculate the unit price as suggested by the referee. \textbf{We make this point clear in Footnote XXX}

\textbf{EN MEME TEMPS: Si on écarte de l'analyse les flux qui sont passés à la fois par bateau et par air, en ne gardant que les flux qui ont fait l'un ou l'autre, ne peut-on pas se dire que la quantité (pas mode-spécifique) a totalement voyage par vessel (si le flux indique qqch pour ves-val) ou par avion (s'il indique qqch pour air-val)? Et à ce moment là, on peut répondre ``vraiment'' à sa critique?} \textbf{GD20200708 Sauf que du coup, on change le sample. Il faudrait vérifier si c’est de beaucoup ou pas}

At a more fundamental level, one might argue that considering the unit price as value/weight is relevant in our setting where we seek to identify transport costs. For instance, it makes sense that the shipment costs of cars do depend not only on the quantity of cars exported, but on the weight it makes, which is related to the volume it takes in the plane or the vessel. \textbf{Do we have evidence, even anecdotal, on this point? Is it an argument we want to make, See what Lashkaripour says about this point also.}

\textit{Calculating the unit price as Value/Weight presents the authors
with an additional endogenity problem. To elaborate, let $\omega_{ik} = Weight/Quantity$
denote the unit weight of the goods in observation $ik$. Also let $\widehat{p}_{ik} = Value/Quantity$
(unlike what the authors assume) denote standard definition of price}. As a consequence, making the link with the fas price we consider in the paper: $\widetilde{p}_{ik} = \widehat{p}_{ik} / \omega_{ik}$.

\textit{This paper is essentially estimating the following equation}:

\begin{equation*}
\ln\left(\frac{p_{ik}}{\widetilde{p}_{ik}}-1 \right)= \ln \left(\tau_{i} \times \tau_{s(k)} -1 +\frac{t_{i} + t_{s(k)}}{\widehat{p}_{ik}/\omega_{ik}} \right) + \epsilon_{ik}
\end{equation*}

\textit{instead of estimating}:


\begin{equation*}
\ln\left(\frac{p_{ik}}{\widetilde{p}_{ik}}-1 \right)= \ln \left(\tau_{i} \times \tau_{s(k)}-1 +\frac{t_{i} + t_{s(k)}}{\widehat{p}_{ik}} \right) + \epsilon_{ik}
\end{equation*}

\textbf{GD20200708 Pourquoi n’introduit-il pas $\widehat{p}$ à gauche ?}
\textit{There is evidence that (i) $\omega_{ik}$ varies significantly within narrowly-defined product
categories, and (ii) $\omega_{ik}$ is negatively correlated with transport costs. So the
way the authors are calculating unit prices and estimating the model creates a
new (but avoidable) source of endogeneity.}


\textbf{Je ne comprends pas bien le point (i)}. We thank the referee for pointing out this potential source of bias in our estimates. If we cannot follow the referee's advice be replacing weight by quantity for data availability reasons, we can address his/her concern regarding this as source of endogeneity bias by instrumenting the fas price. \textbf{il ne faudrait pas dans ce cas mettre de lagged prices, sinon on reintroduit du bruit. Non? Mais alors, on n'explique plus grand chose... }
\textbf{GD20200708 Moi non plus je ne comprends pas très bien}

\subsection{Critique 3: Big Picture Implications}

\textit{My third critique concerns the lack of an exciting punchline. The fact that
composition effects have not countervailed the reduction in pure transport costs
(at least not as much as previously believed) is an interesting but minor observation.
Does this observation revise our understanding of say the gains from
trade? Does it shed new light on a puzzle many people are thinking about?
One crude suggestion is to see how the reduction in the industry-specific cost
terms is related to the industry-level trade elasticities. If the composition effects
favor low-elasticity industries, the findings in the paper may have first-order
implications for the gains from trade.
Another suggestion is to dig deeper into the relative rate at which additive
and multiplicative transport costs have declined over time. Since additive transport
costs favor rich (high-quality exporting) countries, the disproportionally
greater reduction in additive costs can perhaps explain the rise of low-income
exporter as documented by Hanson (2012, JEP).}


\textbf{Comments about that}

\paragraph{Piste 1} Consider his/her first suggestion. What is the idea? First, the link between gains from trade and trade-elasticity. Am I correct in saying that gains from trade are higher when trade is about low-elasticity goods? The idea being, if national goods are low substitutes, then here are larger gains from trading them. So, if composition effects (trade shifts between sectors) favor low-elasticity sectors, then one might expect high gains from trade. Hummels: trade composition effects matter as they partially offset the reduction in pure transport costs for both air and vessel. Over time, tendency to trade goods more costly to export everything else equal. So, if the referee's assertion is right (trade composition effects favor low-elasticity sectors), then increasing gains from trade. But, we disagree with Hummel's findings (his way of measuring things is inappropriate) and find that trade composition effects do not matter much, at least for air (for vessel, the composition effects matter more, but by amplifying the reduction in transport costs, ie towards goods that are less costly to export). Meaning that gains from trade are purely due to reduction in transport costs at the sector, but not from switch towards low-elasticity sectors. Identify one source of gains from trade, not two.

What to do with this? When composition effects do matter (ocean freight), do they favor low-elasticity industries? In which case, on top of the reduction in transport costs per se (which has welfare consequences as well), one might expect additional gains from these composition effects. When they don't, this mitigates the gains from trade that could be expected from the picture given by Hummels.

\textbf{GD20200708 Hum... Je crois que je comprends le point dur referee et je suis d’accord avec lui. Le souci, c’est que de toutes les manières, je ne sais pas si cela favorise les high ou les low elasticity. Mais on pourrait voir.}

\paragraph{Piste 2} What is at stake? Show that how $\beta$ = the relative share of additive costs has evolved over time. Good idea as this is really what differentiates us from Hummels, model a varying $\beta$ over time. Specifically, how methodology lets the $\beta$ vary over the three time-sector-partner country dimension. Show how $\beta$ varies over time everything else equal? Hopefully, goes in the referee's direction (decreases over time) with possibly a role in the understanding in the ``big picture'' of trade patterns?

\paragraph{Piste 3} \textbf{GD20200708 On sait que les additive costs sont plus distorsifs que les multiplicative costs, puisqu’ils modifient les prix relatifs. Donc le fait qu’ils sont une partie croissante des costs signifie qu’on sur-estime l’augmentation des gains de welfare liés à la baisse des coûts de transport}

\appendix


\section{Clarifying some technical points \label{app:technical_points}}


\subsection{A note on prices}

We have three prices: the (US) consumer price, say $p^{c}_{ik}$, the ``import'' or  cif price (at the entrance of the US) $p_{ik}$ and the fas price (at the export gate in the origin country) $\widetilde{p}_{ik}$, denoting $i$ the origin country, $k$ the product at the 8 or 5 digit level, and reasoning on a yearly basis, and $s(k)$ the 3-digit classification the product $k$ belongs to. Further, if we denote $\tau^d_{ik}$ the duty tax rate paid when the good crosses the US border, then we have:

\begin{eqnarray*}
&&p^c_{ik} = (1+\tau^d_{is(k)})p_{ik} \\
with && p_{ik}  = \tau_{is(k)} \widetilde{p}_{ik} +t_{is(k)}
\end{eqnarray*}


\subsection{Deriving the share of additive costs in total transport costs \label{app:interpret_beta}}


With $i$ the origin country, $k$ the product category at the HS6 level (and reasoning at the year- transport mode level), $p_{ik}$ the import (cif) price and $\tilde{p}_{ik}$ the export (fas) price, transport costs $f_{ik}\equiv \frac{p_{ik}- \tilde{p}_{ik}}{\tilde{p}_{ik}} $ are written in Hummel's terminology as:


\begin{equation}
f_{ik} = X_{is(k)}\tilde{p}_{ik}^{-\beta_{ik}} \label{eq:Hummels}
\end{equation}

It is trivial to show that:
$$\frac{\partial f_{ik}}{\partial \tilde{p}_{ik}} \frac{\tilde{p}_{ik}}{f_{ik}}= -\beta_{ik}$$

Then
\begin{itemize}
\item If $\beta_{ik} = 0$, then $p_{ik} = (1+X_{is(k)})\tilde{p}_{ik}$: Only ad-valorem transport costs
\item If $\beta_{ik} = 1$, then $p_{ik}=\tilde{p}_{ik}+X_{is(k)}$: Only additive costs
\end{itemize}

This suggests that, the closer $\beta_{ik}$ to 1, the more prevalent additive costs in total transport costs. Can $\beta_{ik}$ be related to the share of additive costs in total transport costs? The answer is positive. To show this clearly, start from:

\begin{eqnarray*}
f_{ik} &=& \frac{p_{ik}-\tilde{p_{ik}}}{\tilde{p}_{ik}}\\
\text{with}:&& p_{ik} = \tau_{is(k)}\tilde{p}_{ik} + t_{is(k)} \\
\end{eqnarray*}

assuming that transport costs decompose in two components, additive transport costs $t_{is(k)}$ and multiplicative costs (ad-valorem) $\tau_{is(k)}$, supposed to be identical in the product-dimension for any $k$ product within the 3-digit classification $s$ it belongs to.

This gives:

$$f_{ik} = \frac{(\tau_{is(k)}-1)\tilde{p}_{ik}+t_{is(k)}}{\tilde{p}_{ik}}$$

From this, we have

\begin{eqnarray*}
\frac{\partial f_{ik}}{\partial \tilde{p}_{ik}} &=& \frac{(\tau_{is(k)}-1)\tilde{p}_{ik} - ((\tau_{is(k)}-1)\tilde{p}_{ik} +t_{is(k)})}{\tilde{p}_{ik}^2} \\
&=& \frac{-t_{is(k)}}{\tilde{p}_{ik}^2}
\end{eqnarray*}

Hence
\begin{eqnarray*}
\frac{\partial f_{ik}}{\partial \tilde{p}_{ik}} \frac{\tilde{p}_{ik}}{f_{ik}}&=& \frac{-t_{is(k)}}{(\tau_{is(k)}-1)\tilde{p}_{ik} +t_{is(k)}}\\
&=& \frac{-\frac{t_{is(k)}}{\tilde{p}_{ik}}}{\tau_{is(k)}-1 +\frac{t_{ik}}{\tilde{p}_{ik}}}
\end{eqnarray*}

This shows that $\beta_{ik}$ can be interpreted as the share of additive costs in total transport costs:

$$\beta_{ik} = \frac{\frac{t_{is(k)}}{\tilde{p}_{ik}}}{\tau_{is(k)}-1 + \frac{t_{is(k)}}{\tilde{p}_{is(k)}}} $$


\subsection{Share vs level}

Starting from the above reasoning, the referee proposes to estimate the elasticity of transport costs with respect to the unit price, equivalently the share of additive costs in total costs. Specifically, the referee suggests to estimate for each industry-country the following regression:

$$\ln f_{ikd}= \beta_{is(k)} \ln \widetilde{p}_{ikd} + \text{Controls}_{ikd} +\epsilon_{ikd}$$

The question is then, how to recover the levels of additive / multiplicative transports costs $t_{is(k)}$ and $\tau_{is(k)}$. Denoting $\widehat{\beta}_{is(k)}$ the estimated $\beta$ from the referee's method for a given sector-country $i,s(k)$, one can solve the following two-equation system:

\begin{eqnarray}
p_{ik} &=& \tau_{is(k)}\widetilde{p}_{ik} +t_{is(k)} \label{eq:system1}\\
\frac{t_{is(k)}}{(\tau_{is(k)}-1)\widetilde{p}_{ik}} &=& \widehat{\beta}_{is(k)}  \label{eq:system2}
\end{eqnarray}

\noindent with $p_{ik}$ and $\widetilde{p}_{ik}$ the cif and fas prices observed in our dataset (conditional on a given year-transport mode). With 2 equations and 2 endogenous variables, the system might be solved. Specifically, with $\widehat{\beta}$ being estimated at the \textit{sector} $s$- country $i$ level, and $\widetilde{p}_{ik}$ at the \textit{product} $k$-country level, this implies that $t$ and $\tau$ are product-country $i,k$ specific (not sector-country $s,k$ specific). \textbf{from this point of view, this is a finer characterization of transport costs, so better from this point of view...}

To show this properly, start from Equation (\ref{eq:system2}) to express additive costs according to:
$$t_ik = \frac{ \widehat{\beta}_{is(k)}}{1- \widehat{\beta}_{is(k)}} \left[\tau_{is(k)}-1\right]\widetilde{p}_{ik} $$

With $\widetilde{p}_{ik} $ being product-country specific, $t_ik$ is necessarily product-country specific ($i,k$, not $i,s(k)$). Plugging this into Equation (\ref{eq:system1}):

$$p_{ik} = \tau_{is(k)} \widetilde{p}_{ik} + \frac{ \widehat{\beta}_{is(k)}}{1- \widehat{\beta}_{is(k)}} \left[\tau_{is(k)}-1\right]\widetilde{p}_{ik} $$

Solving for $\tau$, one finally obtains the solutions for the ad-valorem and additive transport costs components:

\begin{eqnarray}
\tau_{ik} & =& (1-\widehat{\beta}_{is(k)}) \frac{p_{ik}}{\widetilde{p}_{ik} } + \widehat{\beta}_{is(k)} \label{eq:tau_method_ref1}\\
t_ik &=& \frac{ \widehat{\beta}_{is(k)}}{1- \widehat{\beta}_{is(k)}} \left[\tau_{is(k)}-1\right]\widetilde{p}_{ik} \label{eq:t_method_ref1}
\end{eqnarray}


Two points can be made. First, we can check that if $\widehat{\beta}_{is(k)}=0$, then $p_{ik} = \tau_{ik} \widetilde{p}_{ik}$ (Equation \ref{eq:tau_method_ref1}) and $t_{ik} = 0$ (Equation \ref{eq:t_method_ref2}), all transport costs are ad-valorem. Conversely, if $\widehat{\beta}_{is(k)}=1$, then $p_{ik} = \widetilde{p}_{ik} + t_{ik} $ (Equation \ref{eq:tau_method_ref2}) and $\tau_{ik} = 0$ (Equation \ref{eq:t_method_ref1}). Second, and most importantly, the method allows to recover the levels of both transport costs components at the product-country level $i,k$, even though the share of additive costs in total costs is estimated at the sector-country level.



\section{More details on the instrumentation strategy}

\subsection{Specification of the first stage equation}


The question we deal with, is the functional form of the first stage equation, where we instrument the fas price with duties as suggested by Referee 1. The idea is that firms might react to changes in duty tax rates which have nothing to do with transport costs changes. In this respect, considering the predicted part of the fas price related to tax duty is likely to solve potential endogeneity biases.

Denoting $i$ the origin country, $k$ the product at the 5 digit level, and reasoning on a yearly basis, if we assume that the fas price $\widetilde{p}_{ik}$ decomposes in two components, say $\bar{p}_{ik}$ the ``firm-specific'' price (related to its cost and pricing strategy) and $\tau^d_{ik}$ the tax duty (out of the firm's hands) according to :

\begin{equation}
\widetilde{p}_{ikt} = (1+\tau^d_{is(k)t})^\alpha \left(\bar{p}_{ikt}\right)^\beta \label{eq:link_fas_duty}
\end{equation}

\noindent with $s(k)$ the 3-digit classification as a function of the 5-digit product classification.

\subsubsection{First-stage equation in first difference \label{ssec:first_diff}}

One option to specify the functional form of the first stage is to take the total differential around some reference point at time $t-1$, with $\Delta $ the difference operator ($\Delta\widetilde{p}_{ikt} = \widetilde{p}_{ikt} - \widetilde{p}_{ikt-1}$ and so on):
\begin{eqnarray*}
&&\Delta \widetilde{p}_{ikt} = \beta \bar{p}_{ikt-1}^{\beta-1}(1+\tau^d_{is(k)t-1})\Delta \bar{p}_{ikt} + \alpha \bar{p}^\beta_{ikt-1} (1+\tau^d_{ikt-1})^{\alpha-1}\Delta \tau^d_{is(k)t}  \\
\Leftrightarrow &&\frac{\Delta \widetilde{p}_{ikt}}{\widetilde{p}_{ikt-1}} = \beta \frac{\Delta \bar{p}_{ikt}}{\widetilde{p}_{ikt-1}} \frac{(1+\tau^d_{is(k)t-1})^\alpha \bar{p}_{ikt-1}^\beta}{\bar{p}^\beta_{ikt-1}(1+\tau^d_{is(k)t-1})^\alpha} +\alpha \frac{\Delta \tau^d_{is(k)t-1}}{1+\tau_{is(k)t-1}^d}\frac{(1+\tau^d_{is(k)t-1})^\alpha \bar{p}_{ikt-1}^\beta}{\bar{p}^\beta_{ikt-1}(1+\tau^d_{is(k)t-1})^\alpha} \end{eqnarray*}

leading to:
\begin{equation}
\frac{\Delta \widetilde{p}_{ikt}}{\widetilde{p}_{ikt-1}} =  \beta \frac{\Delta \bar{p}_{ikt}}{\bar{p}_{ikt-1}} +\alpha\frac{\Delta \tau^d_{is(k)t}}{1+\tau_{ikt-1}^d} \label{eq:firststage_Deltalog}
\end{equation}

The intuition behind the Referee 1's endogeneity concerns, is that we need to eliminate from the fas price, any endogenous component that might me related to transport costs. To do so, the referee suggests that we instrument the export price by tariff rates. Put it in plain words, this suggests to run the first-stage regression based on Equation (\ref{eq:firststage_Deltalog}) according to:

$$\frac{\Delta \widetilde{p}_{ikt}}{\widetilde{p}_{ikt-1}} = \alpha \frac{\Delta \tau^d_{is(k)t}}{1+\tau_{is(k)t-1}^d} + \gamma_{i} +\gamma_{k}+\epsilon_{ik}$$

Or equivalently, taking logs:
$$\Delta \log \widetilde{p}_{ikt}= \alpha\frac{\Delta \tau^d_{is(k)t}}{1+\tau_{is(k)t-1}^d} +\gamma_{i} +\gamma_{k}+\epsilon_{ikt}$$


\noindent with the LHS being the growth rate of the fas price (between $t$ and $t-1$), the first term in the RHS the change in duty tax rates, the second and third terms fixed effects to capture changes in the ``firm-specific price'' $\bar{p}_{ik}$, $\epsilon_{ik}$ being the residual. Notice though that the structure of fixed effects should remain consistent between the first and the second stages. This implies to rather consider the following functional form of the first-stage equation:

\begin{equation}
\Delta \log \widetilde{p}_{ikt} = \alpha\frac{\Delta \tau^d_{is(k)t}}{1+\tau_{is(k)t-1}^d} +\gamma_{i} +\gamma_{s}+\epsilon_{ikt} \label{eq:firststage_Deltalog}
\end{equation}

If this reasoning is correct,
\begin{itemize}
\item we should take as predicted value only the predicted price without the fixed effects:
$$\left(\frac{\Delta \widetilde{p}_{ikt}}{\widetilde{p}_{ikt-1}}\right)^{IV} = \widehat{\alpha}\frac{\Delta \tau^d_{is(k)t}}{1+\tau_{is(k)t-1}^d} $$
\item and we should expect a value of the coefficient $\widehat{\alpha}$ between $\Delta \log \widetilde{p}_{ikt}$ and $\frac{\Delta \tau^d_{is(k)t}}{1+\tau_{is(k)t-1}^d} $ between $-1$ and $0$, depending on the degree of ``pricing-to-market'' of firms.
    \begin{itemize}
    \item[-] $\alpha = 0$ corresponds to the case where the firm does not adjust its fas price to the change in tax duty, that would cancel out the influence of the tax duty change on the price paid by the US consumers; this rather corresponds to small firms which do not have enough market power to manipulate their prices following changes in international competition;
    \item[-] $\alpha <0$ corresponds to the degree of``pricing-to-market'' as the firm offsets the impact of the tax change of the price paid by the final consumer by adjusting her producer price in the opposite direction of the tax change,
    \item[-] with $\alpha = -1$ being the extreme case of ``full PTM'' where the firm fully compensates the tax duty change. As shown by Berman, Martin, Mayer (REStats 2012), this is more likely larger firms.
    \end{itemize}
\end{itemize}

Once this is done, we can rebuild the instrumented fas price in level (at time $t$):

\begin{eqnarray*}
\widetilde{p}^{IV}_{ikt} = \widetilde{p}_{ikt-1}\left[1 + \left(\frac{\Delta \widetilde{p}_{ikt}}{\widetilde{p}_{ikt-1}}\right)^{IV}\right]
\end{eqnarray*}

with $\widehat{\widetilde{p}}_{ikt}$ the instrumented fas price at time $t$, for product $k$ from country $i$; $\widetilde{p}_{ikt-1}$ the observed lagged fas price for the same $i,k$ couple; and $ \left(\frac{\Delta \widetilde{p}_{ikt}}{\widetilde{p}_{ikt-1}}\right)^{IV}$ the predicted growth rate of the fas price based on duty changes.

\subsection{First-stage equation in level}

Alternatively, one can run the first-stage equation in log level, consistently with the second-stage log-level specification. Further, in accordance with the referee's suggestion, this also implies including the lagged price as instrument. Starting from Equation (\ref{eq:firststage_Deltalog}), we thus run as first-stage specification:


\begin{equation}
\log \widetilde{p}_{ikt} = \alpha\frac{\Delta \tau^d_{is(k)t}}{1+\tau_{is(k)t-1}^d} + \beta \log\widetilde{p}_{ikt-1}  +\gamma_{i} +\gamma_{s}+\epsilon_{ikt} \label{eq:firststage_log}
\end{equation}

Consistently with the cross-section analysis adopted by now, the first-stage equation is estimated on a yearly basis (and by transport mode). We then build the instrumented fas price

\begin{eqnarray*}
&&\log \widetilde{p}_{ikt}^{IV} = \widehat{\alpha}\frac{\Delta \tau^d_{is(k)t}}{1+\tau_{is(k)t-1}^d} + \widehat{\beta}\log\widetilde{p}_{ikt-1} \\
\rightarrow &&\widetilde{p}_{ikt}^{IV}  = \exp^{\log \widetilde{p}_{ikt}^{IV}}
\end{eqnarray*}



\end{document} 