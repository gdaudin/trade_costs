\documentclass[12pt]{article}
%%%%%%%%%%%%%%%%%%%%%%%%%%%%%%%%%%%%%%%%%%%%%%%%%%%%%%%%%%%%%%%%%%%%%%%%%%%%%%%%%%%%%%%%%%%%%%%%%%%%%%%%%%%%%%%%%%%%%%%%%%%%%%%%%%%%%%%%%%%%%%%%%%%%%%%%%%%%%%%%%%%%%%%%%%%%%%%%%%%%%%%%%%%%%%%%%%%%%%%%%%%%%%%%%%%%%%%%%%%%%%%%%%%%%%%%%%%%%%%%%%%%%%%%%%%%
%TCIDATA{OutputFilter=LATEX.DLL}
%TCIDATA{Version=5.50.0.2960}
%TCIDATA{<META NAME="SaveForMode" CONTENT="1">}
%TCIDATA{BibliographyScheme=Manual}
%TCIDATA{Created=Monday, February 14, 2011 11:40:51}
%TCIDATA{LastRevised=Thursday, February 17, 2011 16:51:46}
%TCIDATA{<META NAME="GraphicsSave" CONTENT="32">}
%TCIDATA{<META NAME="DocumentShell" CONTENT="Standard LaTeX\Blank - Standard LaTeX Article">}
%TCIDATA{CSTFile=40 LaTeX article.cst}

\usepackage{graphicx}
\usepackage{setspace}
%\usepackage{hyperref}
\usepackage{geometry}
\usepackage{blindtext}
\geometry{hmargin=3cm,vmargin=2.5cm}
\usepackage{lscape}
\newtheorem{theorem}{Theorem}
\newtheorem{acknowledgement}[theorem]{Acknowledgement}
\newtheorem{algorithm}[theorem]{Algorithm}
\newtheorem{axiom}[theorem]{Axiom}
\newtheorem{case}[theorem]{Case}
\newtheorem{claim}[theorem]{Claim}
\newtheorem{conclusion}[theorem]{Conclusion}
\newtheorem{condition}[theorem]{Condition}
\newtheorem{conjecture}[theorem]{Conjecture}
\newtheorem{corollary}[theorem]{Corollary}
\newtheorem{criterion}[theorem]{Criterion}
\newtheorem{definition}[theorem]{Definition}
\newtheorem{example}[theorem]{Example}
\newtheorem{exercise}[theorem]{Exercise}
\newtheorem{lemma}[theorem]{Lemma}
\newtheorem{notation}[theorem]{Notation}
\newtheorem{problem}[theorem]{Problem}
\newtheorem{proposition}[theorem]{Proposition}
\newtheorem{remark}[theorem]{Remark}
\newtheorem{solution}[theorem]{Solution}
\newtheorem{summary}[theorem]{Summary}
\newenvironment{proof}[1][Proof]{\noindent\textbf{#1.} }{\ \rule{0.5em}{0.5em}}

\setlength{\parindent}{0pt}

\newcommand*\sepline{%
  \begin{center}
    \rule[1ex]{.5\textwidth}{.5pt}
  \end{center}}
\newcommand*\sepstars{%
  \begin{center}
    $\star\star\star$
  \end{center}}

\begin{document}


\noindent \textbf{Manuscript ID JOEG-2019-027, ``International Transport costs:\\New Findings from modeling additive costs'': revised revision}

\bigskip
\bigskip

\hfill Paris, \today

\bigskip
\noindent Dear Pr. Kerr, dear Bill, \\
\\
\noindent \noindent We would like to sincerely thank you for the opportunity to resubmit our paper, and for your patience in waiting the outcome. We are very grateful for the deadline extensions you kindly granted us. Obviously, the CoViD-19 crisis and the various consecutive lockdowns made the access to our offices difficult in several occurrences, sometimes for several months. \medskip

However, the major source of delays lies in our attempts to obtain additional data to accommodate a very important concern by Referee \#1 (see below). We ordered this new data from the Census Bureau at the end of December 2019. However, some of the years we requested (2000 and 2001, as well as years before 1998) are not on their servers: these data are back at their office which they have not been able to access since Spring 2020 apparently, due to the ongoing health situation. Since then, the situation has never evolved, and we are still currently waiting for these missing years. We can of course make available all our e-mail exchanges with Jeremy Sanchez, our contact at the Census Bureau. Until very recently, Mr. Sanchez did not gave us any clear ``yes or no'' answer to our requests concerning a possible return in his DC office (closed because of the CoViD 19 restrictions), which would have allowed the delivery of the data. In his last e-mail, he confirmed that access to the additional data was impossible.\medskip

Considering the lack of credible perspective to access this additional data in a sufficiently close/foreseeable future, we completed the revision with the following data: a full dataset at the SITC5 level, completed with systematic robustness checks (see the list at the end of this letter) based on a HS10 dataset over the 2005-2019 period.\footnote{We uncovered some mismatches between the original Hummels' dataset (whose finest degree of classification is at the 5-digit level, and which we use until 2004) and the Census data with detailed information at the HS10 level for the years 1997-1999 and 2001-2004 which we could access to. Since 2005 onwards, we use original Census data with perfect match between the 5- and 10-classification degrees. Accordingly, we run our robustness check to aggregation over this period 2005-2019 (Section 3.3.3).} These new estimates do not provide any substantial change to our benchmark estimates, and it seems reasonable to consider that additional checks including years from the 1980s and the 1990s would not dramatically change this outcome. \medskip

To get now more precisely to the revisions undertaken, we thank the referees for their various comments which led to substantial improvements of the paper. We enclose detailed answers to the referees. Below we detail how we handled two major points raised by both referees, before listing the other major changes brought to the paper.

\section{Two Common Concerns: Estimation Method and Big Picture}
Both referees questioned (i) our empirical methodology, especially our choice of a non-linear estimator and (ii) the lack of a ``big picture'' in the paper, i.e. of a deeper analysis of the implications of our results. Below we detail how we answered their concerns:\medskip

\subsection{Estimation Method}
Both referees challenged our choice of a non-linear estimator, suggesting alternative functional forms that could be estimated with standard linear estimators. We do acknowledge that the initial version of the paper did not make a clear enough case for the non-linear estimator. We now explain in details, both in the paper and in our answers to referees, the reasons why a non-linear estimator is unavoidable in our case. This is due to the restrictions imposed \textit{ex-ante} on parameters, i.e. $\tau \geq 1$ and $t\geq 0$. These restrictions are required because we know that both types of transport costs are positive and we cannot neglect this characteristic. By construction we cannot have $p_{ikt} < \widetilde{p}_{ikt}$. Without these restrictions, standard linear, least squares estimates deliver aberrant values with mediocre quality-of-fit. Taking into account this constraint necessarily brings up a functional form requiring a non-linear estimation. We did our best to make the point clearer in the revised version (see section 2.2, more specifically p. 8). \medskip

That said, resorting to non-linear estimations requires some assumptions due to the computational burden. The revised version devotes much more space to explicit tests of robustness of the empirical specification. On top of the robustness to the separability assumption (already in the initially submitted version), we also assess the robustness to potential endogeneity and to the level of aggregation (Sections 3.3.2 and 3.3.3) - see the list of modifications in section 2 from this letter. \medskip

Finally, we also point in our answers that implementing the estimation methodology suggested by Referee 1 would imply non-negligible costs. Firstly, this would leave us much less data to exploit, partly  because of the already mentioned CoViD issues - we would lose half of the years in the sample. Secondly, we show in details in our response that our method yields a more precise estimation of transport costs than the referee's method. \medskip



\subsection{Big Picture}
Both referees asked us to shed light on the implications of additive costs (the ``Big Picture''). The revised version answers this concern in two points. \medskip

First, we maintain the analysis of the transport costs time trends (previously [Section 3], now Section 4); but we made our best to make this section easier to read, leaving the technical aspects in the Appendix (Section C). As such, we hope that the main message of this Section is easier to understand. \medskip

Second, and probably more importantly, we now emphasize the ``Big Picture'' implications of additive costs on theoretical grounds. Both referees suggested various, interesting paths to explore. Obviously, all of them could not be handled in a single paper. We have decided to focus on the welfare implications of our results.  Through the lens of the Melitz's (2003) model amended to integrate additive costs, we analyze the welfare gains that derive from the reduction in international transport costs that we have estimated (in Section 3 of the revised version), to quantify the extra welfare gains attributable to the reduction of the additive component. This analysis is conducted in a new Section 5. We emphasize that the theoretical implications of additive costs have not been fully exhausted in Irarrazabal et al. (2015), Hornok and Koren (2015a,b), Alessandria et al. (2010), or Sorensen (2014). Specifically, we use our empirical results to investigate the welfare consequences of the reduction in international transport costs observed over 1974-2019, contrasting two cases: when this reduction is achieved through a combined reduction in the ad-valorem and the additive components (as quantified by our empirical model) or solely through the ad-valorem cost (as modelled in standard New Trade theoretical approaches). To the best of our knowledge, this is the first time that such an assessment is preformed based on long-time, detailed trade data.
\medskip

Overall, we believe these results add a substantial contribution to the paper : we are able to provide a quantitative assessment of the welfare gains induced by the decrease in both types (additive and multiplicative) of costs over a 45-year period covering the hyper-globalization moment, and to highlight the respective part of each component in the determination of these gains. %Note also that not only the inclusion of additive costs in the underlying framework generates large welfare differences, but also that the latter are probably a lower bound of the welfare variations induced by changes in additive \emph{trade} costs, larger than the sole transport costs.


\section{List of major revisions in the new version.}

Below we list the main other revisions implemented to answer referees' concerns:

\begin{enumerate}


\item \emph{Endogeneity issues}: Section 3.3.2 and Appendix B.3 provide detailed developments addressing concerns expressed by Referee 1 concerning potential endogeneity in our estimates. Relying on instrumental variables suggested by Referee 1, we provide first- and second-stage estimates, based on both the 5-digit-product level and the more disaggregated HS10 level for sensitivity checks. Figure 4 on page 17 in the paper reports our benchmark estimates by transport mode, together with their instrumented counterparts. In all cases, these estimates are very similar. This suggests that our benchmark estimates do not suffer any substantial biases arising from endogeneity concerns.

\item \emph{Aggregation issues}: Referee 1 pointed `` \emph{The original annual Census data reports trade at the origin country-HS10 product-district level of aggregation, whereas the authors are aggregating up the data even further to the origin country-HS5 industry-year level. Such an aggregation comes with strong implicit assumptions and sacrifices a lot of useful variation in the data}.'' Firstly, note these data are available only from 1989, whereas our full sample starts in 1974. Secondly, as already mentioned at the beginning of this letter, it was only possible to get access to the years 1997-1999 and 2001-2019, due to the COVID situation. Considering the importance of having a long time span for our empirical exercise, especially in the context of the welfare analysis, we decided to stick to our initial 5-digit database, and to provide systematic robustness estimates based on the HS10 data available. These results are now reported in section 3.3.3 in the revised version. In particular, Table 3 and Figure 5 show the results are similar when using 10-digit products sand 5-digit products.

\item \emph{Calculation of unit prices}: Referee 1 is concerned by the fact that we calculate unit price as Value/Weight, rather than Value/Quantity. We extensively explain why we decided against a switch from per-kg price to per-unit prices in our answer to the referee and the Online Appendix (which has been extended with a section on this issue). To sum up in a few words, first, there are data availability issues. Second, the estimation is even more computationally intensive. Third, we believe that the study of per-kg prices adds relevant information to the study of international transport costs, both because actual shipping costs are expressed in a per-weight/volume basis and because it allows us to integrate a larger share of trade flows in our analysis.

\end{enumerate}

Altogether, we believe that the paper is much clearer and easier to read. The robustness exercises included in the main text focus on the elements that were the most problematic to the referees. The paper's relevance to big picture debates is now more straightforward. We hope that you will be pleased by the outcome, and we sincerely apologize again for the delay.

\bigskip

Yours sincerely,

\bigskip

\hfill Jérôme Héricourt

\hfill Professor of Economics

\hfill University of Lille (LEM-CNRS) and CEPII

\bigskip
\bigskip
\bigskip
\sepstars







\end{document}

