\documentclass[a4paper,11pt]{article}

%\usepackage[pdftex]{graphicx}
\usepackage{amsmath}
%\usepackage[latin1]{inputenc}
%\usepackage{hyperref}
%\usepackage[T1]{fontenc}
%\usepackage[utf8]{inputenc}
%%GD Quand je mets usepackage[utf8]{inputenc} je n'arrive plus à compiler la biblio
%J'ai essayé de débuger : je n'y suis pas arrivé
\usepackage{rotating}
\usepackage{setspace}
\usepackage{lscape}
\usepackage[round]{natbib}
\usepackage{multirow}
\usepackage{rotating}
\usepackage{vmargin}
\usepackage{epstopdf}
\usepackage{hyperref}
\usepackage{float}
\usepackage{caption}
\usepackage[tableposition=top]{caption}
\usepackage{amsfonts}
\usepackage{bbold}
\usepackage{bbm}
\usepackage[flushleft]{threeparttable}
%\usepackage{bbold}
%\usepackage[T1]{fontenc}
% JH : impossible de compiler avec le package bbold, remplacé par amsfonts
% LP : impossible de compiler avec le package bbold, remplacé par bbm
\usepackage{upgreek}
\usepackage{comment}
\includecomment{commentGD}
\usepackage[draft]{todonotes}
%\usepackage[disable]{todonotes}
%%GDPour cacher les notes, \usepackage[disable]{todonotes}

\usepackage{setspace}

%\usepackage[nolists, figuresfirst]{endfloat}
%\usepackage{sidefloat}

\setlength{\abovecaptionskip}{-8pt}

%\usepackage[usenames]{color}
%\definecolor{grey}{rgb}{0.35,0.35,0.35}
%\definecolor{webdarkblue}{rgb}{0,0,0.4}
%\definecolor{orange}{rgb}{0.7,0.2,0.05}


%\usepackage[pdfcreator={PDFLaTeX}, pdfproducer={PDFLaTeX}, pdfstartview=FitH, pdfpagemode=UseOutlines, pagebackref=false, colorlinks={true},
%citecolor={webdarkblue}, linkcolor={webdarkblue},
%urlcolor={webdarkblue}]{hyperref}



%\addtolength{\oddsidemargin}{-0.4in}
%\addtolength{\evensidemargin}{-0.4in}
%\addtolength{\textwidth}{0.8in} \addtolength{\topmargin}{-0.85in}
%\addtolength{\textheight}{1.7in}
\renewcommand{\baselinestretch}{1.2}

\newcommand\cites[1]{\citeauthor{#1}'s\ (\citeyear{#1})}

\newcommand\citeh[1]{\citeauthor{#1}'\ (\citeyear{#1})}

\setmarginsrb{3cm}{2cm}{3cm}{2cm}{0,5cm}{0,5cm}{0,3cm}{0,5cm}



% commands

\newcommand{\bi}{\begin{itemize}}
\newcommand{\ei}{\end{itemize}}
\newcommand{\be}{\begin{enumerate}}
\newcommand{\ee}{\end{enumerate}}
\newcommand{\bd}{\begin{description}}
\newcommand{\ed}{\end{description}}
\newcommand{\beqa}{\begin{eqnarray}}
\newcommand{\eeqa}{\end{eqnarray}}
\newcommand{\beq}{\begin{equation}}
\newcommand{\eeq}{\end{equation}}
\newcommand{\bs}{\bigskip}
\newcommand{\A}{$^a$}
\newcommand{\B}{$^b$}
\newcommand{\C}{$^c$}

\renewcommand{\baselinestretch}{1.2}
%\doublespacing

%\linespread{1.6}

%\newcommand{\note}[1]{\footnote{\begin{doublespace}#1\end{doublespace}}}


\begin{document}


\title{\textsc{International Transport costs:\\New Findings from modeling additive costs} \\
Answer to Referee 2}

\author{Guillaume \textsc{Daudin}\thanks{%
Université Paris-Dauphine, PSL University, CNRS, 8007, IRD, 260, LEDa, DIAL, 75016, Paris, France ; email: \url{guillaume.daudin@dauphine.psl.eu}}  \qquad J\'{e}r\^{o}me \textsc{H\'{e}ricourt} \thanks{Universit\'{e} de Lille - LEM-CNRS UMR 9221, France \& CEPII, France; email: \url{jerome.hericourt@univ-lille.fr}}\qquad Lise \textsc{Patureau}\thanks{Corresponding author.
University Paris-Dauphine \& PSL University, LEDa, 75016 Paris, France;  email: \url{lise.patureau@dauphine.psl.eu} } }


\date{November 2021}
 \maketitle
\bigskip

We would like to thank you for your insightful comments. They led us to introduce some
significant changes to the paper that we hope address your concerns. We first give you an overview
of the revision (Section \ref{sec:main_changes}) before answering in detail each of your comments (Section \ref{sec:detailed_answers}). Number of sections, pages and equations mentioned in the text refer to the revised version. When necessary, we refer to number of sections, pages and equations from the submitted version. In this case, they are written into brackets.

\section{Main changes \label{sec:main_changes}}

The structure of the paper has been significantly modified along the revision process, in particular to take into account the referees' comments.
\begin{itemize}
	\item In the submitted version, [Section 2] was devoted to the estimation of international transport costs; specifically, their break-down in two components, additive and ad-valorem. In [Section 3], we investigated the role of additive costs in the decomposition of transport costs time trends, between structural changes and composition effects. [Section 4] was devoted to the robustness analysis relative to the results from both previous sections.
	\item In the revised version, we have substantially strengthened the robustness checks regarding the estimation of international transport costs, with the addition of detailed checks to aggregation and endogeneity. Accordingly, the previously-called [Section 2] has been split in two Sections: Section 2, where we present the data and the estimation strategy; and Section 3, where we present the the results. Section 3 now includes the robustness checks as a final sub-section. For the sake of space, the robustness analysis related to composition effects has been sent to the Appendix (Section C.3).
	\item Both referees asked us to devote more developments to the implications of additive costs (the ``Big Picture''). We do agree that this was not emphasized enough in the submitted version, and we thank the referees for their suggestions on that issue. The revised version answers this concern in two points.
	\begin{enumerate}
		\item We maintain the analysis of the transport costs time trends (previously [Section 3], now Section 4); but we improved this section's readability by leaving the technical aspects in the Appendix (Section C). As such, we hope that the main message of this Section is easier to understand.
		\item Most importantly, we now emphasize the ``Big Picture'' welfare implications of additive costs on theoretical grounds. Through the lens of the \cites{melitz} model amended to integrate additive costs, we analyze the welfare gains deriving from the reduction in international transport costs estimated in Section 3 over 1974-2019. In this regard, we shed light on the welfare variations induced by the ``hyper-globalization'' period. This also allows us to quantify the extra welfare gains attributable to reduction of additive trade costs. This analysis is performed in Section 5.


	\end{enumerate}
	\item To keep the paper (both the main text and its various appendices) within a reasonable number of pages, and stay fully transparent about our results, we have added numerous detailed tables to the Online Appendix.
\end{itemize}

We now answer in detail each of your comments given in italics.

\section{Detailed answers \label{sec:detailed_answers}}

\subsection{Implications/usefulness of the exercise}

\paragraph{``Big Picture''} \textit{[...] The fact that transport costs are both specific and ad-valorem is well known. Also, the theoretical implications of having specific, instead of ad-valorem (iceberg), transport cost when it comes to analyzing the consequences of further reductions in transport costs are by now quite well understood; see e.g. papers by Irarrazabal et al. (2015), Hornok and Koren (2015a,b), Alessandria et al. (2010), or Sorensen (2014). The main contribution of this paper thus appears to be to unravel the temporal evolution / relative importance of the ad-valorem and specific transport cost component. However, besides showing this evolution, the paper is basically silent on the implication(s) of the patterns found. In the conclusion, the authors state that this is left for future research. This was quite disappointing, simply documenting the evolution of these two components leaves the reader (at least me) wondering why he/she should care about this. The authors should aim to be much more clear on what the trends in ad-valorem / specific transport costs mean/imply from the perspective of these recent theoretical contributions/insights.}
\smallskip
\paragraph{Exploring the heterogeneity of our results} \textit{At the very least the authors could/should make use of the fact that the temporal patterns in specific / ad-valorem transport costs is likely to show substantial heterogeneity across goods as well as countries. Exploring this heterogeneity to a fuller extent could already hint at the main drivers (insurance costs / containerization / economies of scale) behind the observed patterns. This could be done by e.g. not only reporting the mean and median of the ad-valorem / specific transport cost components but exploring more fully the heterogeneity in these transport costs components across sectors/exporters. Also, it should be relatively straightforward to provide some analysis of the major correlates with the identified specific and ad-valorem costs (e.g. product value/weight ratio, distance to the exporter, etc.).}
\smallskip

\textbf{Our answer:} We devoted a lot of thought to this point, which directly echoed a similar concern by the other referee. The latter writes: ``\emph{Does this observation revise our understanding of say the gains from trade? Does it shed new light on a puzzle many people are thinking about? [One] suggestion is to dig deeper into the relative rate at which additive and multiplicative transport costs have declined over time.}
''

All in all, your suggestion and the other referee's one pointed toward two possible options for the ``Big Picture'' implications of our exercise: either exploring the compositional country-product dynamics underlying of our results, or focusing on the welfare implications. Both paths were undeniably worth exploring, but could not be followed in a single paper. Therefore, we decided to focus on the second path, i.e. to  offer insights on the welfare implications of our results. In this regard, the new Section 5 ``The role of additive cost: Theoretical insights'' is devoted to a theory-based analysis of the changes to welfare gains implied by the relative variations of additive and multiplicative transport costs over our period of analysis.

Our choice to dig into the welfare implications of our results has two main motivations. Firstly, even if ``\emph{the fact that transport costs are both specific and ad-valorem is well known}'', some recent literature still debates the practical importance of additive costs - for example, \cite{Lashkaripour_JIE2020} finds evidence supporting the ad-valorem assumption for denumerable goods, and consistently finds a very small welfare effect of additive costs within a partial-equilibrium model with perfect competition. Secondly, the implications of additive costs have not been fully exhausted in \citet{Irrazabal_2015}, \cite{Hornok-et-al-RES-2015, Hornok-et-al-JIE-2015}, \cite{Alessandria-et-al-AER-2010} or \citet{sorensen2014}. \cite{Kropf-Saure-JIE-2016} estimate the size and shape of per-shipment costs based on Swiss data. \cite{Alessandria-et-al-AER-2010} and \cite{Hornok-et-al-RES-2015} point out the role of per-shipment costs in generating ``lumpiness'' in international trade transactions. \cite{Hornok-et-al-JIE-2015} focus on administrative costs incurring with every shipment (i.e., additive by nature), and draw welfare conclusions regarding the variations of these costs. We complement these various outcome by providing a long-run characterization of both additive and multiplicative component of transports costs. Based on these empirical results, we investigate the welfare consequences of the reduction in international transport costs observed over 1974-2019. In this regard, we shed light on the welfare variations induced by the ``hyper-globalization'' period, as well as the disproportionate part of additive transport costs in determining those welfare variations.\smallskip

To this aim, we introduce additive costs in the canonical \citet{melitz} model. The inclusion of additive costs in a \citet{melitz} setting has already been performed in \citet{sorensen2014} and \citet{Irrazabal_2015}. However, \citet{sorensen2014} exclusively performs a theoretical analysis, without any quantitative exercise. In addition to a partial equilibrium extension of \citet{melitz}, \citet{Irrazabal_2015} do perform a quantitative simulation to assess the welfare variations induced by the reduction of additive costs (vs ad-valorem costs). For this, they rely on a calibration for transport costs based on a single-year estimation (for the year 2004), from which they depart by assuming an ad-hoc hypothetical reduction of trade costs. In contrast, our own exercise relies on a several decades of US data, allowing us to highlight the welfare alterations induced over time by the relative dynamics of additive and multiplicative costs, based on the values actually observed in the data over the period 1974-2019, in all their dimensions (additive and ad-valorem).\smallskip

More precisely, we use the estimates underlying results reported in Section 3 concerning multiplicative and additive costs, to implement several comparative statics exercises to investigate the different welfare consequences of alterations to multiplicative and additive costs. In addition, we also assess how the latter results are distorted by changes in sunk costs of exports, $f_{x}$. To that end, we adjust the share of exporting firms in the US based on \citet{BEJK-AER-03} and \citet{Lincoln_McCallum2018}. In our preferred exercise, we quantify the welfare gains deriving from the reduction in variable transport costs as documented in Section 3, relying on a combined reduction in each additive/ad-valorem component in a context of fixed export cost reduction. We compare setting this with the case where the reduction of total transport costs  is solely attributed to a decrease in ad-valorem costs.

Table 5 in Section 5 of the paper reports the results of these various comparative statics exercises for Air and Vessel transport modes. Specifically, we report the welfare effect of change in total transport costs decomposed in its two dimensions (additive and ad-valorem), both in absolute and in relative terms. We have two major insights. First, using the reduction of transport costs in vessel (air),  welfare gains are around 50\% (14\%) higher when this reduction is achieved through the estimated reduction in additive and ad-valorem costs. Second, the decrease in export sunk costs (i.e., increase in the share of exporting firms) proportionally amplifies the gains from decreasing variable costs, with again an additional premium coming from the decrease of additive costs.

Overall, these results add a substantial contribution to the paper : we are able to provide a quantitative assessment of the welfare gains induced by the decrease in both types (additive and multiplicative) of costs over a 45-year period covering the hyper-globalization moment, and to highlight the respective part of each component in the determination of these gains.
%Note also that not only the inclusion of additive costs in the underlying framework generates large welfare differences, but also that the latter are probably a lower bound of the welfare variations induced by changes in additive \emph{trade} costs, larger than the sole transport costs.





%\textbf{WHAT DO WE DO ABOUT THAT? We provide an analysis by sector about the time trends. Do more than that: Regress the additive costs/ ad-valorem costs $t_{is}$, $\tau_{is}$ on distance from the US? correlation btw each component and the sector/country fixed effect ?}

\subsection{Empirical strategy}

\textit{The empirical strategy used to unravel the specific and ad-valorem component in international
transport costs is quite elaborate. [...] To be more specific, why not simply estimate equation (1) while plugging
in equation (3) and (4) for $\tau$ and $t$ respectively? Doing this yields the following estimation equation (\ref{eq:ref2_basis}) in
levels (I have added a time subscript t as well as an additive error term $\epsilon$):}

\begin{equation}
p_{ikt}=\tau_{it}\widetilde{p}_{ikt}+ \tau_{kt}\widetilde{p}_{ikt} + t_{it} + t_{kt} + \epsilon_{ikt}   \label{eq:ref2_basis}
\end{equation}

\textit{I do not see why the authors need the complicated non-linear transformations to do what they aim
for. Estimating Equation (\ref{eq:ref2}) should be relatively easy, for one it is linear, but it would also not require the
inclusion of any country-sector fixed effects.}

\begin{equation}
p_{ikt}= \sum_i \sum_t \alpha_{it}^\tau \mathbbm{1}_{it}\widetilde{p}_{ikt} +  \sum_k \sum_t \alpha_{kt}^\tau \mathbbm{1}_{kt}\widetilde{p}_{ikt} + \sum_i \sum_t \alpha_{it}^t \mathbbm{1}_{it} + \sum_k \sum_t \alpha_{it}^t\mathbbm{1}_{kt} \label{eq:ref2}
\end{equation}


\textbf{Our answer:} As highlighted by the referee, our estimated equation imposes to use non-linear estimation methods, such as Non-Linear Least Squares. We agree that resorting on a linear form as the one suggested by the referee would substantially alleviate the computational burden. However, it would have some important drawbacks.

First, it can be noticed that the equation suggested by the referee does not stand in accordance with the assumption we borrow from \cite{Irrazabal_2015}, that ad-valorem costs is decomposed in its two country/sector dimensions multiplicatively (i.e. approximating $\tau_{is(k)}$ by the product $\tau_i\times \tau_{s(k)}$).

However, even with another formulation, such as the equations (\ref{eq:ref2_basis}) and (\ref{eq:ref2}) suggested by the referee, we would still be constrained to resort to non-linear estimators. This is due to the restrictions imposed \textit{ex-ante} on parameters, i.e. $\tau \geq 1$ and $t\geq 0$, the latter meaning simply we constrain both types of transport costs to be non-negative, as by construction we cannot have $p_{ikt} < \widetilde{p}_{ikt}$. Without these restrictions, standard linear, least squares estimates deliver aberrant values with mediocre quality-of-fit. Taking into account $p_{ikt} < \widetilde{p}_{ikt}$ also implies that the error term should be always positive, which we ensure by specifying the error term such that:

\begin{equation*}
\frac{p_{ik}}{\widetilde{p}_{ik}}-1 =\left(\tau_{i}\times \tau_{k} -1+\frac{t_{i} + t_{k}}{\widetilde{p}_{ik}} \right)\times \exp(\epsilon_{ik})
\end{equation*}
\noindent where $\epsilon_{ik}$ follows a normal law centered on 0.

This specification obviously requires a non-linear estimation. We did our best to make the point clearer in the revised version (see section 2.2, more specifically p. 8). We also devote more time to the robustness of the empirical specification. On top of the robustness to the separability assumption (already in the previously submitted version), we also assess the robustness to potential endogeneity and the level of aggregation (Sections 3.3.2 and 3.3.3). We hope that this new version will convince you that our estimation strategy is both relevant and robust.



\newpage
\bibliographystyle{apalike2}
\bibliography{biblio}





\end{document} 